In this thesis, we have introduced the hierarchical topology representation for solving scheduling problems. We have described how to represent scheduling systems, how to model their complex behaviors, how to evaluate them, and how to optimize them both in an offline and an online setting. Through a series of experiments, we have demonstrated that this representation can be used to solve online scheduling problems across a variety of system and event characteristics, deeming the representation both effective and versatile.

There are several directions in which future research could go. One of them is exploring different feature sets in online scheduling algorithms, as discussed in \ref{sec:online_scheduling_limitations}. Another direction could be to explore different genotype operators for the algorithms which performed the best in our experiments, to further optimize these representations for online scheduling problems. Finally, since all the case studies in this thesis were artificial, it would be interesting to solve real problems using the concepts presented in this thesis. The hierarchical topology representation proved to be highly effective for solving online scheduling problems in a somewhat idealized environment, which indicates that it potentially could be used for solving problems in real environments as well.