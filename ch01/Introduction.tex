Scheduling is the decision-making process which solves the problems of allocating limited resources to tasks over a given time. The model of scheduling used in this thesis has the following setup - a system contains $m$ machines which are organized in some way, a total $k$ jobs need to be processed on the system, and the scheduling in this context entails deciding, at every moment, which job is being processed on which machine, in order to optimize some criteria. 

Scheduling problems are interesting optimization problems because they are hard to solve. For most non-trivial setups, an exhaustive approach is not applicable. Instead, heuristic approaches are used. In this thesis, our goal will be to solve online scheduling problems, where the exact sequence of jobs which needs to be processed is not known in advance, and instead the information about a job becomes known only when it enters the system. We will solve such problems using hyperheuristics, which are heuristic approaches used to find good ways to solve a problem.

Chapter \ref{sec:theoretical_background} of this thesis covers the theoretical background required for solving online scheduling problems. Chapter \ref{sec:topology_model} introduces a hierarchial topology model, which is a versatile model capable of representing a wide range of scheduling systems. Chapter \ref{sec:events_model} introduces the events model, which will be used for modelling complex behaviors such as machine breakdowns, preemptions, prerequisites, machine buffers with a limited size, setups and batch processing. Chapter \ref{sec:evaluation_model} introduces the evaluation model for the hierarchial topology representation. Chapter \ref{sec:optimization_model} covers the concepts which allow building modular optimization algorithms, and which will be used in all optimization algorithms throughout this thesis. Chapter \ref{sec:offline_scheduling_model} explains how the hierarchial topology representation can be used for solving offline scheduling problems. In such problems, data about jobs is completely known in advance, which in theory makes the optimization easier to perform. While offline scheduling is not a central concept for this thesis, understanding it will be a great stepping stone to understanding online scheduling. Chapter \ref{sec:online_scheduling_model} explains how the hierarchial topology representation can be used for solving online scheduling problems. Chapter \ref{sec:experiments_and_results} covers several online scheduling case studies, and includes hyperparameter tuning, experimental setups, results and examples of generated heuristics. On the surface, these case studies encompass problems which are quite different one from another. But they will all be solved using the same scheduling framework, as a testament to the versatility and complexity of the hierarchial topology representation. Finally, chapter \ref{sec:conclusion} covers the conclusion of the thesis.