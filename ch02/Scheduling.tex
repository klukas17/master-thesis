\section{Scheduling}
\label{sec:scheduling}

\subsection{Model of scheduling}
\label{sec:model_of_scheduling}
Scheduling is a decision-making process used to allocate resources to tasks over a given time, with the goal of optimizing one or more objectives \citep{pinedo2016scheduling}. This model can be applied to a many different domains. For example, the resources can represent machines in a workshop, and the tasks can be operations in a production process. In another example, the resources can be crews at a construction site, and the tasks can be stages in a construction process. In one more example, the resources can be processing units in a computing environment, and the tasks can be the executions of computer programs. The objectives can be the minimization of the completion time of the last task, or the minimization of the number of tasks completed after their deadline, or any other criterion used for evaluating the schedule.

Following the notation described in \citep{pinedo2016scheduling}, a scheduling problem can be described by a triplet $\alpha | \beta | \gamma$. The field $\alpha$ describes the machine environment. The field $\beta$ describes processing characteristics and constraints. The field $\gamma$ describes the objective to be optimized. All these fields and their possible values will be described in detail in chapters \hyperref[sec:topology_model]{3}, \hyperref[sec:events_model]{4} and \hyperref[sec:evaluation_model]{5} respectively, when we discuss the system proposed in this thesis, which is an implementation of this model of scheduling.

To illustrate the inherent complexity of solving a scheduling problem, we will consider the problem $1|r_j|L_{max}$. In this problem, the $\alpha$ parameter is $1$, which means the system contains a single machine. The $\beta$ parameter is $r_j$, which means that the job $j$ becomes available at time $r_j$, where $r$ is the symbol for release date. The $\gamma$ is an objective function called maximum lateness function, where the maximum of $L_j = C_j - d_j$ for every job $j$ is chosen, $C_j$ is the time that the job exits the system, and $d_j$ is the due time when the job was expected to exit the system. This problem is NP-hard, as proven in \citep{pinedo2016scheduling}. Compared to problems we will use for experiments in chapter \ref{sec:experiments_and_results}, it is a relatively simple problem, but nevertheless, in terms of the computation theory, it is a hard problem. Due to the complexity which arises when dealing with scheduling, heuristic optimization approaches play a central role in scheduling optimization problems.

\subsection{Online and offline scheduling}
\label{sec:offline_and_online_scheduling}
Scheduling problems can be categorized in two main categories, offline and online scheduling. In an offline scheduling problem, all problem data is known prior to solving the problem. In this context, the term problem data refers to information about the number of tasks, and the characteristics of each task. In an online scheduling problem, the problem data is not known in advance, and instead it becomes available only when the system is executing. The number of tasks may not be known in advance, and the characteristics of each task become available only when the task becomes active and available to solve \citep{pinedo2016scheduling}. In this thesis, we will discuss both offline and online scheduling.

\subsection{Representations for scheduling problems}
\label{sec:representations_for_scheduling_problems}
There are several types of machine environments which can be expressed using the model of machines and jobs, and for each of them, distinct representations have been proposed \citep{werner2013survey}. For flow shops, where each job has the same route through the system, a permutation of jobs as representation of the solution has been proposed. For job shops, where each job has its own route, a similar approach has been used, with each machine having its own separate permutation. Finally, for open shops, where no limitations are imposed on job routes, several representations have been proposed, with one of them being a string encoding for the schedule construction.

There are also several systems which can be thought of as implementations of this model. One of them is Lekin \citep{lekin}, developed as an educational tool, and used for offline scheduling across several workspace environments: single machine, parallel machines, flow shop, flexible flow shop, job shop and flexible job shop.

In the subsequent chapters, a new system of representation will be proposed, called \textit{hierarchical topology representation}. This system is a faithful implementation of the model of machines and jobs, and it supports both offline and online scheduling.